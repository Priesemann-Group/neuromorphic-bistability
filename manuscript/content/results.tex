
\begin{figure*}[ht]
	\centering
	\begin{tikzpicture}
		\node[anchor=north west,inner sep=0pt] (b) at (0,-3.8) {\includegraphics[width=.58\columnwidth]{images/chip.JPG}};
		\node[anchor=north west,inner sep=0pt] (a) at (0.1,0.0) {\tikzset{%
	inh/.style = {%
		draw=nicered,%
		fill=nicered,
		circle,
		inner sep=0pt,
		minimum width=0.3cm,
	},%
	exc/.style = {%
		draw=niceblue,%
		fill=niceblue,
		regular polygon,
		regular polygon sides=3,
		minimum width=0.4cm,
		inner sep=0pt,
	},%
}%
\pgfmathsetseed{42}%

\begin{tikzpicture}[
		x=2.3cm,
		y=2.3cm,
		>=stealth,
		line width=1.0\pgflinewidth,
		anchor=center
        ]

	\useasboundingbox (-1.4,0.65) rectangle (0.7,-0.82);

	% network layers
	\foreach \b in {0,1,...,4}{
		\ifthenelse{\b=3}{
			\node[inh] (background \b) at (-1.2,0.5-0.25*\b) {};
		}{
			\node[exc] (background \b) at (-1.2,0.5-0.25*\b) {};
		}
	}

	\pgfmathsetmacro{\N}{9}
	\pgfmathsetmacro{\r}{0.45}
	\coordinate (center) at (0,0);
	\foreach \n in {0,1,...,\N}{
		\ifthenelse{\n=2 \OR \n=5}{
			\node[inh] (hidden \n) at ($({\r*sin(\n*360/(\N+1))},{\r*cos(\n*360/(\N+1))})+(center)$) {};
		}{
			\node[exc] (hidden \n) at ($({\r*sin(\n*360/(\N+1))},{\r*cos(\n*360/(\N+1))})+(center)$) {};
		}
	}

        \begin{scope}[on background layer]
			% connectivity
			\begin{scope}
			\path[scope fading=south] (-0.8,-0.1) rectangle ++(1.6,-2.0);
			\foreach \h in {0,1,...,\N}{
				\foreach \i in {0,1,...,4}{
    	        			\pgfmathparse{rnd}
    	        			\pgfmathsetmacro{\foobar}{\pgfmathresult}
    	        			\ifthenelse{\lengthtest{\foobar pt<0.2 pt}}{
						\ifthenelse{\i=3}{
							\draw[-stealth,nicered] (background \i) -- (hidden \h);
						}{
							\draw[-stealth,niceblue] (background \i) -- (hidden \h);
						}
					}{}
				}
			}
			\end{scope}
			
			\fill[white,opacity=0.7] (0,0) circle (1.5cm);
			\draw[nicegray,semithick] (0,0) circle (1.5cm);

			\foreach \i in {0,1,...,\N}{
				\foreach \j in {0,1,...,\N}{
					\ifthenelse{\equal{\j}{0}}{
						% no self connections
					}{
						\pgfmathparse{rnd}
						\pgfmathsetmacro{\foobar}{\pgfmathresult}
						\ifthenelse{\lengthtest{\foobar pt<0.2 pt}}{
							\pgfmathsetmacro{\bend}{ifthenelse(\j <= \N / 2 + 1,"bend right","bend left"))}
							\pgfmathsetmacro{\target}{int(Mod(\i + \j, \N + 1))}
							\ifthenelse{\i=2 \OR \i=5}{
								\draw[-stealth,nicered] (hidden \i) to[\bend] (hidden \target);
							}{
								\draw[-stealth,niceblue] (hidden \i) to[\bend] (hidden \target);
							}
						}{}
				}
				}
			}
		\end{scope}

	\coordinate (label) at (0.0,-0.77);
	\node[nicegray,anchor=center] at (label -| hidden 0) {\footnotesize Network};
	\node[nicegray,anchor=center] at (label -| background 0) {\footnotesize Input};
\end{tikzpicture}
};
		\node[anchor=north west,inner sep=0pt] (c) at (5.15,0.0) {\input{build/figures/rate_evolution.pgf}};
		\node[anchor=north west,inner sep=0pt] (d) at (5.15,-3.8) {\input{build/figures/final_rate.pgf}};
		\node[anchor=north west,inner sep=0pt] (e) at (9.3,0.0) {\input{build/figures/weight_histogram.pgf}};
		\node[anchor=north west,inner sep=0pt] (f) at (9.3,-3.8) {\input{build/figures/connectivity.pgf}};
		\node[anchor=north west,inner sep=0pt] (g) at (13.45,-3.8) {\input{build/figures/average_weight.pgf}};
		\node at ($(a.north west) + (0.1,-0.2)$) {\textbf{A}};
		\node[text=white] at ($(b.north west) + (0.2,-0.2)$) {\textbf{B}};
		\node at ($(c.north west) + (0.2,-0.2)$) {\textbf{C}};
		\node at ($(d.north west) + (0.2,-0.2)$) {\textbf{D}};
		\node at ($(e.north west) + (0.2,-0.2)$) {\textbf{E}};
		\node at ($(f.north west) + (0.2,-0.2)$) {\textbf{F}};
		\node at ($(g.north west) + (0.2,-0.2)$) {\textbf{G}};
	\end{tikzpicture}

	\caption{%
		\textbf{Reducing the input strength to homeostatically regulated networks of \gls{ei} \gls{lif} neurons strengthens recurrent connections.}
		\textbf{(A)} Illustration of a random network topology with \SI{80}{\percent} excitatory (blue triangles) and \SI{20}{\percent} inhibitory (red circles) neurons.
		\textbf{(B)} Image of the BrainScaleS-2 neuromorphic chip.
		Image taken from \citep{muller_extending_2020}.
		\textbf{(C)} Homeostatic plasticity regulates the population rate $\nu$ close to a target value (dashed line).
		\textbf{(D)} For a broad range of external input rates, $\nu$ approximates the target rate.
		\textbf{(E)} The stochastic homeostatic regulation leads to heterogeneous weight distributions for both, inhibitory and excitatory synapses.
		The counts of excitatory weights exceed the inhibitory ones by a factor of four due to the imposed \gls{ei} ratio.
            \textbf{(F)} The effective connectivity, defined as the percentage of non-zero recurrent synapses ($c_{ij}w^\mathrm{rec}_{ij}\neq 0$), does not saturate at its maximum (dashed line) for decreasing input strengths.
		\textbf{(G)} However, the mean weight increases to compensate for a reduction of input.
	}
	\label{fig:chip}
\end{figure*}

\section{\label{sec:results}Results}

%\subsection{Reducing external input to homeostatically regulated neuromorphic hardware strengthens recurrent connections}
\subsection{Homeostatically regulated neuromorphic hardware compensates lack of external input by strengthening recurrent connections}

% homeostatic regulation
To begin, we verify that the experimental setup --- the neuromorphic chip with homeostatic regulation during development --- reaches a stationary dynamical state with firing rates sufficiently close to the target rate.
Starting from the initial condition of zero recurrent weights ($w^\mathrm{rec}_{ij}=0$), we observe for our chosen parameters a transient relaxation behavior that reaches a stationary firing rate after about \SI{200}{} update iterations, independent of the external input rate $h$ (\cref{fig:chip}C).
Note that for this representation, the firing rate is evaluated over an interval of $T=\SI{100}{\second}$ between iterations, and further averaged over \num{50} network realizations.
One can see that for larger values of $h$ (blue curve) the relaxation is smoother than for lower values of $h$ (red curve).
The stationary firing rates are close to the target rate $\nu^\ast=\SI{10}{\hertz}$ (\cref{fig:chip}D), however, there is a systematic $h$-dependence that presumably originates from the firing rate being a non-linear function of the mean coupling, $\nu(\langle w\rangle)$, as observed in mean-field calculations of \gls{ei} networks~\cite{mastrogiuseppe_intrinsically-generated_2017}. %which might in turn promote unstable activity further enhanced by the limited precision weight dynamics.
%Since, in this work, we do not seek a perfect match of the target rate but are interested in the self-organized collective dynamics of homeostatically regulated networks in general, we conclude that the homeostatic-plasticity rule reliably regulates the neuron firing rates sufficiently close to the target firing rate.
Since we find consistent results for reference computer simulations (see Supplemental Material), we conclude that the experimental setup reliably self-organizes into a stationary dynamics state with neuron firing rates sufficiently close to the target rate.

% reducing input strengthens recurrency
We next investigate how homeostatically regulated \gls{ei} networks compensate a reduction of external input with a strengthening of recurrent connections (\cref{fig:chip}E-G).
In particular, we find that the histograms of both inhibitory as well as excitatory recurrent coupling weights become flatter with decreasing $h$, indicating strong heterogeneity (\cref{fig:chip}E).
%It is noteworthy that the limited range of weight values is an effect of the \SI{6}{\bit} integer weights on the neuromorphic hardware.
Interestingly, the effective connectivity -- the fraction of all physical $K^\mathrm{rec}$ recurrent weights that are not zero -- does not reach its maximum theoretical value of $K^\mathrm{rec}/N = 100 / 512 \approx \SI{20}{\percent}$ (\cref{fig:chip}F).
Instead, it even decreases for low $h$, which is likely a consequence of the strong variability of rates between iterations (cf. \cref{fig:chip}C) that results in large weight changes for the given plasticity rule and does not affect our main conclusions (see Supplemental Material for a milder plasticity rule).
%This assumption is further strengthened by a reduced decline of the effective connectivity for a non stochastic and hence more continuous implementation of homeostatic regulation (\cref{fig:hom_comparison}B).

\begin{figure*}[ht]
	\centering
	\begin{tikzpicture}
		\node[anchor=north west,inner sep=0pt] (x) at (0.94,-3.8) {\import{build/figures/}{colorbar.pgf}};
		\node[anchor=north west,inner sep=0pt] (a) at (0,0) {\input{build/figures/ac.pgf}};
		\node[anchor=north west,inner sep=0pt] (b) at (4.2,0) {\input{build/figures/activity_distribution.pgf}};
		\node[anchor=north west,inner sep=0pt] (c) at (8.5,0) {\input{build/figures/activity.pgf}};
		\node[anchor=north west,inner sep=0pt] (d) at (13.6,0) {\input{build/figures/timescale_comparison.pgf}};
		\node at ($(a.north west) + (0.2,-0.2)$) {\textbf{A}};
		\node at ($(b.north west) + (0.2,-0.2)$) {\textbf{B}};
		\node at ($(c.north west) + (0.2,-0.2)$) {\textbf{C}};
		\node at ($(d.north west) + (0.2,-0.2)$) {\textbf{D}};

		\draw[-latex,ultra thick] ($(c.north east) + (0.2,-0.15)$) -- ($(c.south east) + (0.2,0.8)$) node[midway,xshift=0.2cm,rotate=-90] {Bistability};
	\end{tikzpicture}
	\caption{%
		\textbf{Reducing the input strength increases autocorrelation of network rate through emergent bistability.}
        \mbox{\textbf{(A)} For} low input rates $h$, the population activity exhibits exponentially shaped autocorrelations $C(t')$ with autocorrelation times $\tau_\mathrm{AC}$ significantly exceeding the largest single neuron timescale.
		\textbf{(B)} In this regime, the distribution $P(\nu)$ of the population rate $\nu$ shows a bimodal trend.
		\textbf{(C)} The associated phases of high and low $\nu$ can be fitted by a two state \acrshort{hmm}.
		\textbf{(D)} Based on the transition rates of this \acrshort{hmm}, an equivalent timescale $\tau_\mathrm{HM}$ can be estimated which coincides with $\tau_\mathrm{AC}$ for low $h$.
	}
	\label{fig:time_scale}
\end{figure*}

More important is the observation that, as shown in \cref{fig:chip}G, the mean coupling weights $w^\mathrm{rec}$ increase almost linearly with decreasing input rate.
%In addition to the observed heterogeneity, we clearly find that the mean inhibitory and excitatory weights increase almost linearly with reducing the input rate (\cref{fig:chip}G).
A fit of the form
\begin{equation}\label{eq:w-h}
    \langle w^\mathrm{rec}\rangle(h) = \alpha - \beta h,
\end{equation}
where $\langle.\rangle$ stands for average across synaptic connections over either excitatory or inhibitory populations, yields $\alpha\approx\SI{22.75}{}$ and $\beta\approx\SI{14.23}{}$ for excitatory or $\alpha\approx\SI{26.1}{}$ and $\beta\approx\SI{16.7}{}$ for inhibitory weights.
Hence, a reduction in input rate clearly strengthens the recurrent connections in homeostatically regulated \gls{ei} networks consistent with the theoretical consideration that the loss of external input needs to be compensated by recurrent activity generation in order to maintain a constant firing rate~\cite{zierenberg_homeostatic_2018}.



% E-I balance
In addition to supporting general theoretical arguments, our setup allows us to investigate how our homeostatic self-organization affects the interplay between excitatory and inhibitory neurons.
In fact, it is quite surprising that the mean coupling weights for excitatory and inhibitory weights are so similar (\cref{fig:chip}G), i.e., $\langle w^\mathrm{rec}_\mathrm{inh} \rangle \approx \langle w^\mathrm{rec}_\mathrm{exc}\rangle$, given that each neuron receives four times more input from excitatory than from inhibitory neurons.
Naively, this implies strong excitation dominance in contrast to the expected inhibition dominance required for asynchronous irregular activity~\cite{vreeswijk_chaos_1996,brunel_dynamics_2000} to reproduce experimental single-neuron statistics~\cite{burns_spontaneous_1976, softky_highly_1993, stevens_input_1998, stein_neuronal_2005}.
This outcome can, however, be explained by our symmetric plasticity rule that does not distinguish between excitatory and inhibitory synapses and thereby fosters solutions with $\langle w^\mathrm{rec}_\mathrm{inh} \rangle \approx \langle w^\mathrm{rec}_\mathrm{exc}\rangle$.
For small networks with homogeneous weights (see \cref{sec:phase_diagrams}), the condition $w^\mathrm{rec}_\mathrm{inh} \approx w^\mathrm{rec}_\mathrm{exc}$ turns out to be in the vicinity of a phase transition between regular (high-firing) and irregular (low-firing) dynamics.
Reducing $h$ makes this transition more abrupt and closer to $w^\mathrm{rec}_\mathrm{inh} = w^\mathrm{rec}_\mathrm{exc}$, implying that homeostatic plasticity regulates \gls{ei} networks towards a regular-to-irregular transition when decreasing the external input rate.

% now theoretical prediction of increasing autocorrelation time
\subsection{Homeostatically regulated neuromorphic hardware with low external input generates large autocorrelation times through emergent bistability}

Next, we verify the theoretical prediction~\cite{zierenberg_homeostatic_2018} that a homeostatically regulated system exhibits an increased autocorrelation to compensate for a decreasing external input (\cref{fig:time_scale}).
For this, we consider a network after homeostatic development with fixed weights and evaluate the autocorrelation function of the population firing rate $\nu(t)$ over an interval of $T=\SI{100}{\second}$.
Indeed, the autocorrelation functions, $C(t^\prime)$, show increasingly long tails with decreasing input rate $h$ (\cref{fig:time_scale}A).
Moreover, they are well described by exponential decays, $C(t^\prime)=e^{-t^\prime/\tau_\mathrm{AC}}$, with increasing autocorrelation times $\tau_\mathrm{AC}$ for decreasing $h$ (\cref{fig:time_scale}A inset).
While this general trend has been reported for much smaller neuromorphic systems before~\cite{cramer_control_2020}, the inset of Fig.~\ref{fig:time_scale}A reveals the emergence of two distinct regimes.
For $h>\SI{0.8}{\kilo\hertz}$, we find autocorrelation times to saturate with increasing $h$, suggesting that the uncorrelated Poisson input successfully decorrelates already weakly correlated activity, giving rise to an \textit{input-driven regime}.
In contrast, for $h<\SI{0.8}{\kilo\hertz}$, we find an apparent divergence of $\tau_\mathrm{AC}$ with decreasing $h$, such that this regime is characterized by dominant recurrent activation compensating for the lacking input, which results in increasing autocorrelation times for decreasing $h$, giving rise to a \textit{recurrent-driven regime}.

Surprisingly, we observe that the autocorrelations originate from a bistable population rate (\cref{fig:time_scale}B-D).
Specifically, the distribution $P(\nu)$ changes from unimodal for higher input strengths to bimodal for lower input strengths (\cref{fig:time_scale}B).
The latter suggests that the population rate starts to alternate between two distinct states.
Indeed, close inspection of the time evolution of $\nu(t)$ reveals that for decreasing input strength the population rate switches between a low-rate state and a high-rate state (\cref{fig:time_scale}C), resembling up-and-down states in cortical networks~\cite{wilson_up_2008,stern_spontaneous_1997,cossart_attractor_2003,hidalgo_stochastic_2012}.
Such a switching behavior is reminiscent of a Markov jump process between states of high- and low-firing rates~\cite{ibe_3_2013}, specifically a two-state \gls{hmm}~\cite{rabiner_introduction_1986}.
We fitted a two-state \gls{hmm} to the stationary population rate (discretized in steps of $\Delta t$) and obtained a $2\times 2$ Markov matrix.
%using the Python toolbox hmmlearn~\cite{noauthor_hmmlearnhmmlearn_2022}, which returns a $2\times 2$ Markov matrix for a stationary process.
Since the rate is stationary, the first eigenvalue is one, $\lambda_1=1$, and the second eigenvalue $\lambda_2$ determines how quickly perturbations decay back to the stationary solution.
This is related to the autocorrelation time as $\tau_\mathrm{HM}=-\Delta t/ \log{(\lambda_2)}$.
Indeed, the autocorrelation time of the \gls{hmm} correlates with the autocorrelation time measured from the population rate for small input strengths, where the population rate becomes bistable (\cref{fig:time_scale}D).
This is fundamentally different from the \textit{a-priori} expected close-to-critical fluctuations~\cite{zierenberg_homeostatic_2018}, which would lead to scale-free avalanches~\cite{beggs_neuronal_2003} for small $h$ that we do not observe (see \cref{sec:avalanches}).
We thus conclude that the emergent bistability is the underlying mechanism of the large autocorrelation times observed in the population dynamics of homeostatically regulated \gls{ei} networks of \gls{lif} neurons.






%Instead, the network dynamics are better described as a Markov jump process between states of high- and low-firing rates~\cite{ibe_3_2013}.
%The population rate can be well approximated by a two-state \gls{hmm}~\cite{rabiner_introduction_1986}.
%    We fitted a two-state \gls{hmm} to the population rate (discretized in steps of $\Delta t$) using Python toolbox hmmlearn~\cite{noauthor_hmmlearnhmmlearn_2022}, which returns a $2\times 2$ transition matrix.
%    \textcolor{blue}{TODO: WHY, what is the largest eigenvalue}
%The second-largest eigenvalue of this matrix, $\lambda_2$, is the autocorrelation time of Markov process, $\tau_\mathrm{HM}=-\Delta t/ \log{(\lambda_2)}$.
%Indeed, the autocorrelation time of the \gls{hmm} correlates with the autocorrelation time measured from the population rate for small input strengths, where the population rate becomes bistable (\cref{fig:time_scale}D).
%We thus conclude that the emergent bistability is the underlying mechanism of the large autocorrelation times observed in the population dynamics of homeostatically regulated \gls{ei} networks of \gls{lif} neurons.

\subsection{Computer simulations reveal increasing dynamical barrier of emergent bistability with system size}

Since our neuromorphic hardware only supports networks with up to $512$ \gls{lif} neurons, we use computer simulations to verify the experimental results for increasing network sizes.
In brief, we parametrize the simulations to match the experimental setup and use the Brian2 Python package to solve the model (for details see \cref{sec:methods}).
Indeed, we can reproduce the experimental results with our software implementation: We observe comparable bistable activity with similar autocorrelation functions (see Supplemental Material).
However, while computer simulations in principle allow us to study any system size, they are much less efficient than the neuromorphic emulation.
It is worth noting that for our application, i.e., homeostatically regulating a network of $N=512$ \gls{lif} neurons for \SI{6000}{\second}, the computer simulation on an Intel Xeon E5-2630v4 (roughly $\SI{100 000}{\second}$ at about $\SI{50}{\watt}$) takes $\mathcal{O}(10^4)$ more time and $\mathcal{O}(10^7)$ more energy compared to the corresponding emulation on BrainScaleS-2 (about $\SI{6}{\second}$ at a power budget of \SI{100}{\milli\watt}).


%While on BrainScaleS-2 the emulation of a network with $N=512$ \gls{lif} neurons for \SI{6000}{\second} including homeostatic regulation takes about $\SI{6}{\second}$ at a power budget of \SI{100}{\milli\watt}, a corresponding computer simulation on an Intel Xeon E5-2630v4 amounts to around $\SI{30}{\hour}$ at approximately \SI{50}{\watt}.
%This makes the emulation much faster and about $10^7$ times more energy efficient than the simulation.

\begin{figure}[t]
	\centering
	\begin{tikzpicture}
		\node[anchor=north west,inner sep=0pt] (a) at (0,0) {\input{build/figures/simulation_ac.pgf}};
		\node[anchor=north west,inner sep=0pt] (b) at (4.2,0) {\input{build/figures/simulation_activity_distribution.pgf}};
		\node at ($(a.north west) + (0.2,-0.2)$) {\textbf{A}};
		\node at ($(b.north west) + (0.2,-0.2)$) {\textbf{B}};
	\end{tikzpicture}
	\caption{%
		\textbf{Finite-size scaling of homeostatically regulated \gls{ei} networks with \gls{lif} neurons from computer simulations.}
		\textbf{(A)} Autocorrelation time $\tau_\mathrm{AC}$ as a function of system size $N$ for different external input rates $h$.
        One can see a faster than power-law growth for $h<\SI{0.8}{\kilo\hertz}$, while $\tau_\mathrm{AC}$ seems to saturate on the order of the dominant single-neuron timescale (dashed line) for $h>\SI{0.8}{\kilo\hertz}$.
		\textbf{(B)} Distributions of the population firing rate in windows of $\SI{5}{\milli\second}$ for $h=\SI{0.7}{\kilo\hertz}$ show the bimodal shape remains for increasing $N$.
		The barrier in between high- and low-firing states grows with $N$.
    	}
	\label{fig:sim}
\end{figure}

Having established that the computer simulation reproduces the experimental results, we can study how the measured autocorrelation time $\tau_\mathrm{AC}$ depends on the network size $N$ (\cref{fig:sim}A).
Due to the large computational efforts, we focus on four representative input strengths: a low input strength ($h=\SI{0.7}{\kilo\hertz}$) where we observe bistable activity in the experiment, two medium input strengths ($h=\SI{0.8}{\kilo\hertz}$ and $h=\SI{0.9}{\kilo\hertz}$) near the onset of bistability, and a high input strength ($h=\SI{1.0}{\kilo\hertz}$) where the network does not exhibit bistability.
Only for $h=\SI{0.7}{\kilo\hertz}$, we observe an exponential increase in autocorrelation time with system size that exceeds $\SI{1}{s}$ for the largest $N$.
Instead, at $h=\SI{0.8}{\kilo\hertz}$ the autocorrelation time appears to grow as a power law, while for even larger values of $h$ the $\tau_\mathrm{AC}$ start to saturate on the order of the dominant single-neuron timescale (dashed line).
Our numerical results further corroborate the classification into two distinct regimes: A recurrent-driven regime for low input strength with large emergent autocorrelations and the input-driven regime for high input strength with vanishing autocorrelations.
%The transition between these regimes can be studied in a more systematic way employing finite-size scaling analyses.

To further investigate the origin of the emergent autocorrelations, we study the shape of the probability distribution of local population rates $\nu(t)$ as a function of network size (\cref{fig:sim}B).
We observe that for low input strength, the bimodal distribution becomes more pronounced with increasing suppression of intermediate population-rate values.
%TODO: What is with the other two cases...SI!!! but plot
One can relate the suppression of intermediate rates to a \textit{dynamical barrier} by interpreting the time course of the instantaneous population rates as a trajectory of the dynamical system in the potential $V(\nu) = -\log P(\nu)$.
This barrier would be analogous to the activation energy in an Arrhenius-type equation, i.e., $r\propto e^{-\Delta V/T}$, such that for a given level of fluctuation $T$ the rate $r$ to transition between low- and high-firing-rate regimes is lowered for increasing barriers $\Delta V$.
Since the height of this dynamical barrier increases with $N$, this explains the increasing autocorrelation time with system size.


\begin{figure}[t]
	\begin{tikzpicture}
		\node[anchor=north west,inner sep=0pt] (a) at (0,0)  {\input{build/figures/theory_potential.pgf}};
		\node[anchor=north west,inner sep=0pt] (b) at (4.2,0){\input{build/figures/theory_activity.pgf}};
		\node at ($(a.north west) + (0.2,-0.2)$) {\textbf{A}};
		\node at ($(b.north west) + (0.2,-0.2)$) {\textbf{B}};
	\end{tikzpicture}
	\caption{%
        \textbf{Mean-field theory of emergent bistability upon reducing input to homeostatically regulated recurrent network.}
        Our mean-field theory describes the temporal evolution of the fraction of active neurons, $\rho$, with meta-stable solutions given by the minimum of the potential, \cref{eq:potential}.
        \textbf{(A)} For suitable parameters ($\tau_\mathrm{MF}=10$, $\alpha=30$, $\beta=15$, $b=25$, $\sigma=50$, $N=512$), the potential exhibits a single minimum for large $h$ but two minima for small $h$.
        \textbf{(B)} Numeric evaluation of the corresponding stochastic mean-field equation ($\Delta t=10^{-7}$) shows fluctuating dynamics for large $h$ and emergent bistability for low $h$.
        }
	\label{fig:theory}
\end{figure}



\subsection{Mean-field theory of emergent bistability from fluctuation-induced switching between metastable active and quiescent states}

To qualitatively explain how bistability can emerge in recurrent network with heterogeneous weights, we construct a simple mean-field theory based on the time evolution of a fraction of active neurons at a given time $t$, $\rho(t)$, which can be considered a proxy of the population rate $\nu(t)$ up to some factor.
Let us consider a general mean-field ansatz
\begin{align}
	\dot{\rho}(t) = &-\tau_\mathrm{MF}\rho(t) \nonumber \\
		&+h  (1-\rho(t)) + \left( 1-\rho(t) \right) \left[ \omega_1\rho(t) + \omega_2\rho^2(t)+\dots\right]\, ,
\end{align}
%
where the first term describes the spontaneous decay of activity in the absence of inputs with some characteristic time scale
$\tau_\mathrm{MF}$, the term proportional to $h$ represents external input that can only activate inactive neurons (hence the $(1-\rho)$ factor), and the last term represent the gain function that describes recurrent activations, expanded in power-series of the activity.
Here, the coefficients of expansion ($\omega_1$, $\omega_2$, ...) are an effective representation of the full coupling matrix $w^\mathrm{rec}_{ij}$ (with $\omega_1$ proportional to the mean synaptic strength).
The mean-field equation can be rewritten in a more compact form by grouping-up terms with different powers of the activity,
\begin{equation}\label{eq:MF_compact}
    \dot{\rho}(t) = h - a\rho(t) -b\rho^2(t) + \dots \, ,
\end{equation}
%
where $a=\tau_\mathrm{MF}+h-\omega_1$ and $b=\omega_1-\omega_2>0$ to ensure stability.
It is important to notice that this mean-field equation assumes infinitely large network sizes, $N\to\infty$, for which additional noise terms vanish.

%finite
To describe finite networks one needs to introduce an additional stochastic term to the mean-field \cref{eq:MF_compact} that accounts for demographic fluctuations.
Demographic fluctuations are characteristic of systems with an absorbing or quiescent state~\cite{henkel_absorbing_2008}, where fluctuations of the total number of active units around some mean $N\rho(t)$ are expected to have a standard deviation that scales with $\sqrt{N\rho(t)}$ as a consequence of the central-limit theorem.
For the fraction of active nodes in a finite network, we then obtain to leading order in system size
\begin{equation}\label{eq:MF_noise}
    \dot{\rho}(t) = h - a\rho(t) -b\rho^2(t) + \sqrt{\rho(t)/N}\eta(t)\, ,
\end{equation}
where $\eta(t)$ is Gaussian white noise with zero mean and variance $\sigma^2$.
This (Ito) Langevin equation can be expressed as a Fokker Planck equation, with the steady-state solution~\cite{munoz_nature_1998} (see also SI)
\begin{equation}
    P(\rho) = \mathcal{N}\exp\left\{-\frac{2N}{\sigma^2}V(\rho)\right\}\, ,
\end{equation}
a normalization constant $\mathcal{N}$, and the potential
\begin{align}\label{eq:potential}
    V(\rho) = \left(\frac{\sigma^2}{2N} - h\right)\ln\rho + a\rho + \frac{b}{2}\rho^2\, .
\end{align}
This potential $V(\rho)$ can either have a single (formally diverging) minimum at $\rho=0$ (unimodal activity distribution), or it can have two local minima (bistable activity distribution).
%Specifically, we can find from the condition $\rho\,dV/d\rho=(\frac{\sigma^2}{2N}-h) + a\rho +b\rho^2=0$ extrema at $\rho_\pm = \left(-a\pm\sqrt{a^2-4b(\frac{\sigma^2}{2N}-h)}\right)/2b$ such that a bistable solution occurs when
The condition for extrema of the potential $V$ imply that a bistable solution occurs when
%\begin{equation}\label{eq:condition-bistable}
    $a^2-4b(\frac{\sigma^2}{2N}-h)>0$.
%\end{equation}
With the additional conditions for a positive density, i.e., $\rho>0$, as well as a positive slope at $\rho=0$, i.e., $\rho^2\frac{d^2V}{d\rho^2}(0)=(\frac{\sigma^2}{2N}-h)>0$, we expect to observe bistable dynamics for
\begin{equation}\label{eq:condition-bistable}
    a<-2\sqrt{b\left(\frac{\sigma^2}{2N}-h\right)}<0.
\end{equation}

% homeostatic regulation
To incorporate the effect of training recurrent weights with homeostatic regulation, we recall our empirically obtained anticorrelation, $\langle w \rangle = \alpha - \beta h$, upon homeostatic training (\cref{fig:chip}G).
In our mean-field theory, \cref{eq:MF_compact}, we assume this to dominantly affect $a=\tau_\mathrm{MF} + h - \omega_1 \approx \tau_\mathrm{MF}  - \alpha + h(1+\beta)$ and make the common assumption that $b=\omega_1-\omega_2$ is constant up to higher-order effects.
Inserting $a$ into \cref{eq:condition-bistable}, we find that ---for suitable parameters--- lowering $h$ can indeed induce a transition from a unimodal to a bimodal potential (\cref{fig:theory}).
%Then the condition for a bistable solution, \cref{eq:condition-bistable}, becomes $\tau_\mathrm{MF}-\alpha +h(1+\beta) < -2\sqrt{b(\frac{\sigma^2}{2N}-h)}$ and we find that ---for suitable parameters--- lowering $h$ does indeed induce a transition from a unimodal to a bimodal potential (\cref{fig:theory}).

% stochastic simulations
The $h$-dependent transition from unimodal to bimodal can be visualized by numerically evaluating the mean-filed model (\cref{fig:theory}B).
The numerical integration of \cref{eq:MF_noise} is straightforward~\cite{dornic_integration_2005}, but needs special care to avoid running into the domain of negative numbers due to numerical imprecisions (see Appendix~\ref{sec:appendix_meanfield}).
The resulting trajectories show typical demographic fluctuations for higher inputs and bistable activity for lower input.
Since the involved parameters are not easily related in an explicit way to the experiment, this theoretical result is a qualitative explanation of the observed effect and all parameters are in arbitrary units.

Our mean-field theory implies that emergent bistable population activity can be rationalized as a fluctuation-induced switching between a metastable active and a quiescent phase.
For a system with an absorbing to active non-equilibrium phase transition for vanishing input, we find that finite-size fluctuations are responsible for a metastable active state (high rate) and external fluctuations lead to a metastable quiescent state (low rate).
To transit from one state to another, the system needs to overcome a dynamical barrier, where the transition from high-to-low rate requires demographic noise, whereas the transition from low-to-high rate requires external noise.
