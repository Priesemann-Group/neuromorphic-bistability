\begin{abstract}

A fruitful approach towards neuromorphic computing is to mimic mechanisms of the brain in physical devices, which led to successful replication of neuron-like dynamics and learning in the past.
However, there remains a large set of neural self-organization mechanisms whose role for neuromorphic computing has yet to be explored.
One such mechanism is homeostatic plasticity, which has recently been proposed to play a key role in shaping network dynamics and correlations.
Here, we study --- from a statistical-physics point of view -- the emergent collective dynamics in a homeostatically-regulated neuromorphic device that emulates a network of excitatory and inhibitory \acrlong{lif} neurons.
Importantly, homeostatic plasticity is only active during the training stage and results in a heterogeneous weight distribution that we fix during the analysis stage.
We verify the theoretical prediction that reducing the external input in a homeostatically regulated neural network increases temporal correlations, measuring autocorrelation times exceeding $\SI{500}{\milli\second}$, despite single-neuron timescales of only $\SI{20}{\milli\second}$, in both experiments on neuromorphic hardware and in computer simulations.
However, unlike theoretically predicted near-critical fluctuations, we find that temporal correlations can originate from an emergent bistability.
We identify this bistability as a fluctuation-induced stochastic switching between metastable active and quiescent states in the vicinity of a non-equilibrium phase transition.
Our results thereby constitute a new, complementary mechanism for emergent autocorrelations in networks of spiking neurons with implications for future developments in neuromorphic computing.


%A fruitful approach towards neuromorphic computing is to mimic mechanisms of the brain in physical devices, which led to successful replication of neuron-like dynamics and learning.
%However, there remains a large set of neural self-organization mechanisms whose role for neuromorphic computing is yet to be explored.
%One such mechanisms is homeostatic plasticity, which has recently been proposed to play a key role in shaping network dynamics and correlations.
%Here, we study the emergent collective dynamics in a homeostatic neuromorphic device, emulating a network of excitatory and inhibitory \acrlong{lif} neurons, from a statistical-physics point of view.
%Importantly, homeostatic plasticity is only active during the training stage, resuling in a heterogeneous but fixed weight distribution during the analysis stage.
%We reproduce the theoretical prediction that lowering the external input increases temporal correlations, measuring in both experiments on neuromorphic hardware and in computer simulations autocorrelation times exceeding $\SI{500}{\milli\second}$ despite single-neuron timescales of only $\SI{20}{\milli\second}$, which is compatible with those observed in the living brain.
%However, different to the theoretically predicted close-to-critical fluctuations, we find that temporal correlations can originate from an emergent bistability.
%%A unique feature of neuromorphic computing is that memory is an implicit part of processing through traces of past information in the system's collective dynamics.
%%The extent of memory about past inputs is commonly quantified by the autocorrelation time of collective dynamics.
%%Based on past experimental evidence, a potential explanation for the underlying autocorrelations are close-to-critical fluctuations.
%%Here, we show for self-organized networks of excitatory and inhibitory \acrlong{lif} neurons that autocorrelations can originate from emergent bistability upon reducing external input strength.
%We identify this bistability as a fluctuation-induced stochastic switching between metastable active and quiescent states in the vicinity of a non-equilibrium phase transition.
%%
%%Our results provide  verification of biologically compatible autocorrelation times in networks of \acrlong{lif} neurons, which here are not generated by close-to-critical fluctuations but by emergent bistability in homeostatically regulated networks.
%Our results thereby constitute a new, complementary mechanism for emergent autocorrelations in networks of spiking neurons, with implications for biological and artificial networks

%There is rising experimental evidence that networks of spiking neurons can exhibit emergent autocorrelations that are associated with their information processing capacities.
%However, autocorrelation times as large as in experiments have so far not been reproduced for networks of \acrlong{lif} neurons.
%Here, we demonstrate that networks of excitatory and inhibitory \acrlong{lif} neurons, with homeostatic regulation during development, can exhibit increasing autocorrelation times upon decreasing input connections.
%In our experiments on neuromorphic hardware and in computer simulations, we observe autocorrelation times exceeding $\SI{500}{\milli\second}$ despite single-neuron timescales of only $\SI{20}{\milli\second}$.
%These autocorrelations can be attributed to emergent bistable population activity as a result of fluctuation-induced stochastic switching between metastable active and quiescent states in the vicinity of a non-equilibrium phase transition.
%Our results present
%    (i) a new mechanism for emergent autocorrelations in networks of spiking neurons,
%    (ii) a new perspective on the origin of so-called up-and-down states, and
%    (iii) a general paradigm of fluctuation-induced bistability for driven systems with absorbing states.
%This is a first step towards a fundamental understanding of emergent collective dynamics for neuromorphic computing with self-organized circuits.
\end{abstract}
