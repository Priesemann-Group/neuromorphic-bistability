
\begin{figure*}[t]
	\centering
	\scalebox{0.82}{
	\begin{tikzpicture}
		\node[anchor=north west,inner sep=0pt] (a) at (0,0) {\input{build/figures/si_emul_final_rate.pgf}};
		\node[anchor=north west,inner sep=0pt] (b) at (0,-4.1) {\input{build/figures/si_sim_offset_final_rate.pgf}};
		\node[anchor=north west,inner sep=0pt] (c) at (0,-8.2) {\input{build/figures/si_sim_final_rate.pgf}};

		\node[anchor=north west,inner sep=0pt] (d) at (4.1,0) {\input{build/figures/si_emul_weight_histogram.pgf}};
		\node[anchor=north west,inner sep=0pt] (e) at (4.1,-4.1) {\input{build/figures/si_sim_offset_weight_histogram.pgf}};
		\node[anchor=north west,inner sep=0pt] (f) at (4.1,-8.2) {\input{build/figures/si_sim_weight_histogram.pgf}};

		\node[anchor=north west,inner sep=0pt] (g) at (12.4,0) {\input{build/figures/si_emul_connectivity.pgf}};
		\node[anchor=north west,inner sep=0pt] (h) at (12.4,-4.1) {\input{build/figures/si_sim_offset_connectivity.pgf}};
		\node[anchor=north west,inner sep=0pt] (i) at (12.4,-8.2) {\input{build/figures/si_sim_connectivity.pgf}};

		\node[anchor=north west,inner sep=0pt] (j) at (16.5,0) {\input{build/figures/si_emul_average_weight.pgf}};
		\node[anchor=north west,inner sep=0pt] (k) at (16.5,-4.1) {\input{build/figures/si_sim_offset_average_weight.pgf}};
		\node[anchor=north west,inner sep=0pt] (l) at (16.5,-8.2) {\input{build/figures/si_sim_average_weight.pgf}};

		\node[rotate=90] (label0) at ($(j.east) + (0.7, 0.0)$) {\normalsize\bfseries Emulation};
		\node[rotate=90] (label1) at ($(k.east) + (0.7, 0.0)$) {\normalsize\bfseries Simulation (offset)};
		\node[rotate=90] (label2) at ($(l.east) + (0.7, 0.0)$) {\normalsize\bfseries Simulation};

		\node at ($(a.north west) + (0.2,-0.2)$) {\textbf{A}};
		\node at ($(b.north west) + (0.2,-0.2)$) {\textbf{B}};
		\node at ($(c.north west) + (0.2,-0.2)$) {\textbf{C}};
		\node at ($(d.north west) + (0.2,-0.2)$) {\textbf{D}};
		\node at ($(e.north west) + (0.2,-0.2)$) {\textbf{E}};
		\node at ($(f.north west) + (0.2,-0.2)$) {\textbf{F}};
		\node at ($(g.north west) + (0.2,-0.2)$) {\textbf{G}};
		\node at ($(h.north west) + (0.2,-0.2)$) {\textbf{H}};
		\node at ($(i.north west) + (0.2,-0.2)$) {\textbf{I}};
		\node at ($(j.north west) + (0.2,-0.2)$) {\textbf{J}};
		\node at ($(k.north west) + (0.2,-0.2)$) {\textbf{K}};
		\node at ($(l.north west) + (0.2,-0.2)$) {\textbf{L}};

		\begin{pgfonlayer}{background}
			\node[inner sep=6pt,rounded corners,fill=nicegreen!20,fit=(a) (label0)] (emulation) {};
			\node[inner sep=6pt,rounded corners,fill=nicered!20,fit=(b) (label1)] (simulation_offset) {};
			\node[inner sep=6pt,rounded corners,fill=niceblue!20,fit=(c) (label2)] (simulation) {};
		\end{pgfonlayer}
	\end{tikzpicture}
}
	\caption{%
		\textbf{Software simulations validate the neuromorphic emulation on BrainScaleS-2.}
		The panels \textbf{(A)}, \textbf{(D)}, \textbf{(G)} and \textbf{(J)} show the emulation results already presented in \cref{main-fig:chip} in the main part of this manuscript.
		The panels \textbf{(B)}, \textbf{(E)}, \textbf{(H)} and \textbf{(K)} illustrate the results of corresponding software simulations incorporating the amplitude offset within the synaptic currents present on the neuromorphic system, whereas \textbf{(C)}, \textbf{(F)}, \textbf{(I)} and \textbf{(L)} depict the simulation data acquired for ideal dynamics and hence without offset.
	}
	\label{fig:simulation_comparison_homeostasis}
\end{figure*}

\section{Validation of the analog emulation \label{sec:validation}}

Equivalent software simulations validate the main results on bistability in \gls{ei} networks of \gls{lif} neurons emulated on the analog BrainScaleS-2 system.
In the following, we very the emulation by first comparing to an accurate software simulation incorporating the offset in the synaptic current shown in \cref{main-fig:parametrization} G and H.
Most importantly, our results do not depend on the latter and persist for ideal synaptic dynamics.

The firing rates (\cref{fig:simulation_comparison_homeostasis} A, B and C), the histograms of synaptic weights (\cref{fig:simulation_comparison_homeostasis} D, E and F), the effective connectivity (\cref{fig:simulation_comparison_homeostasis} G, H and I) as well as the average synaptic weights (\cref{fig:simulation_comparison_homeostasis} J, K and L) are comparable for emulation and both simulation conditions.
However, for high input strengths, the firing rate $\nu$ of simulated networks exceeds the target rate.
Likewise, the effective connectivity and the average synaptic weights are slightly higher on the BrainScaleS-2 chip compared to the simulation.
This hints towards an overestimation of the \gls{psp} height on hardware which is most likely due to saturation effects promoted by the analog implementation.

\begin{figure*}[t]
	\centering
	\begin{tikzpicture}
		\node[anchor=north west,inner sep=0pt] (a) at (0,0) {\input{build/figures/si_emul_ac.pgf}};
		\node[anchor=north west,inner sep=0pt] (b) at (0,-4.1) {\input{build/figures/si_sim_offset_ac.pgf}};
		\node[anchor=north west,inner sep=0pt] (c) at (0,-8.2) {\input{build/figures/si_sim_ac.pgf}};

		\node[anchor=north west,inner sep=0pt] (d) at (4.0,0) {\input{build/figures/si_emul_activity_distribution.pgf}};
		\node[anchor=north west,inner sep=0pt] (e) at (4.0,-4.1) {\input{build/figures/si_sim_offset_activity_distribution.pgf}};
		\node[anchor=north west,inner sep=0pt] (f) at (4.0,-8.2) {\input{build/figures/si_sim_activity_distribution.pgf}};

		\node[anchor=north west,inner sep=0pt] (g) at (8.0,0) {\input{build/figures/si_emul_activity.pgf}};
		\node[anchor=north west,inner sep=0pt] (h) at (8.0,-4.1) {\input{build/figures/si_sim_offset_activity.pgf}};
		\node[anchor=north west,inner sep=0pt] (i) at (8.0,-8.2) {\input{build/figures/si_sim_activity.pgf}};

		\node[anchor=north west,inner sep=0pt] (j) at (12.4,0) {\input{build/figures/si_emul_timescale_comparison.pgf}};
		\node[anchor=north west,inner sep=0pt] (k) at (12.4,-4.1) {\input{build/figures/si_sim_offset_timescale_comparison.pgf}};
		\node[anchor=north west,inner sep=0pt] (l) at (12.4,-8.2) {\input{build/figures/si_sim_timescale_comparison.pgf}};

		\node[anchor=north west,inner sep=0pt] (x) at (0.95,-12.5) {\scalebox{0.903}{\import{build/figures/}{colorbar.pgf}}};

		\draw[-latex,ultra thick] ($(g.north east) + (0.2,-0.15)$) -- ($(g.south east) + (0.2,0.8)$) node[midway,xshift=0.2cm,rotate=-90] {Bistability};
		\draw[-latex,ultra thick] ($(h.north east) + (0.2,-0.15)$) -- ($(h.south east) + (0.2,0.8)$) node[midway,xshift=0.2cm,rotate=-90] {Bistability};
		\draw[-latex,ultra thick] ($(i.north east) + (0.2,-0.15)$) -- ($(i.south east) + (0.2,0.8)$) node[midway,xshift=0.2cm,rotate=-90] {Bistability};

		\node[rotate=90] (label0) at ($(j.east) + (0.6, -0.1)$) {\vphantom{(}\small\bfseries Emulation};
		\node[rotate=90] (label1) at ($(k.east) + (0.6, -0.1)$) {\vphantom{(}\small\bfseries Simulation (offset)};
		\node[rotate=90] (label2) at ($(l.east) + (0.6, -0.1)$) {\vphantom{(}\small\bfseries Simulation};

		\node at ($(a.north west) + (0.2,-0.2)$) {\textbf{A}};
		\node at ($(b.north west) + (0.2,-0.2)$) {\textbf{B}};
		\node at ($(c.north west) + (0.2,-0.2)$) {\textbf{C}};
		\node at ($(d.north west) + (0.2,-0.2)$) {\textbf{D}};
		\node at ($(e.north west) + (0.2,-0.2)$) {\textbf{E}};
		\node at ($(f.north west) + (0.2,-0.2)$) {\textbf{F}};
		\node at ($(g.north west) + (0.2,-0.2)$) {\textbf{G}};
		\node at ($(h.north west) + (0.2,-0.2)$) {\textbf{H}};
		\node at ($(i.north west) + (0.2,-0.2)$) {\textbf{I}};
		\node at ($(j.north west) + (0.2,-0.2)$) {\textbf{J}};
		\node at ($(k.north west) + (0.2,-0.2)$) {\textbf{K}};
		\node at ($(l.north west) + (0.2,-0.2)$) {\textbf{L}};

		\begin{pgfonlayer}{background}
			\node[inner sep=6pt,rounded corners,fill=nicegreen!20,fit=(a) (label0)] (emulation) {};
			\node[inner sep=6pt,rounded corners,fill=nicered!20,fit=(b) (label1)] (simulation_offset) {};
			\node[inner sep=6pt,rounded corners,fill=niceblue!20,fit=(c) (label2)] (simulation) {};
		\end{pgfonlayer}
	\end{tikzpicture}
	\caption{%
		\textbf{Equivalent software simulations validate the emergence of bistable population activity for low inputs strengths and lead to comparable timescales.}
		The panels \textbf{(A)}, \textbf{(D)}, (\textbf{G}) and \textbf{(J)} show the emulation results already presented in \cref{main-fig:time_scale} in the main part of this manuscript.
		The panels \textbf{(B)}, \textbf{(E)}, \textbf{(H)} and \textbf{(K)} illustrate the results of corresponding software simulations incorporating the amplitude offset within the synaptic currents present on the neuromorphic system, whereas \textbf{(C)}, \textbf{(F)}, \textbf{(I)} and \textbf{(L)} depict the simulation data acquired for ideal dynamics and hence without offset.
	}
	\label{fig:simulation_comparison}
\end{figure*}

The bistable population activity and the associated emerging autocorrelation persist in simulation.
More specifically, the autocorrelation functions as well as the estimated autocorrelation times of simulation and emulation mostly coincide (\cref{fig:simulation_comparison} A, B and C), the distribution of the population activity shows a similar bimodal trend (\cref{fig:simulation_comparison} D to I) and the autocorrelation times are well described by the \gls{hmm} (\cref{fig:simulation_comparison} J, K and L).
Hence, we conclude that the bistable behaviour as well as the associated emergent autocorrelations are indeed a result of the model dynamics.

\begin{figure*}[t]
	\centering
	\begin{tikzpicture}
		\node[anchor=north west,inner sep=0pt] (a) at (0,0) {\input{build/figures/si_hom_final_rate.pgf}};
		\node[anchor=north west,inner sep=0pt] (b) at (4.2,0) {\input{build/figures/si_hom_connectivity.pgf}};
		\node[anchor=north west,inner sep=0pt] (c) at (0,-4.1) {\input{build/figures/si_hom_weight_histogram.pgf}};
		\node[anchor=north west,inner sep=0pt] (d) at (8.4,0) {\input{build/figures/si_hom_average_weight.pgf}};
		\node[anchor=north west,inner sep=0pt] (e) at (8.4,-4.1) {\input{build/figures/si_hom_ac.pgf}};
		\node[anchor=north west,inner sep=0pt] (f) at (12.6,0) {\input{build/figures/si_hom_activity_distribution.pgf}};
		\node[anchor=north west,inner sep=0pt] (g) at (12.6,-4.1) {\input{build/figures/si_hom_activity.pgf}};
		\node at ($(a.north west) + (0.2,-0.2)$) {\textbf{A}};
		\node at ($(b.north west) + (0.2,-0.2)$) {\textbf{B}};
		\node at ($(c.north west) + (0.2,-0.2)$) {\textbf{C}};
		\node at ($(d.north west) + (0.2,-0.2)$) {\textbf{D}};
		\node at ($(e.north west) + (0.2,-0.2)$) {\textbf{E}};
		\node at ($(f.north west) + (0.2,-0.2)$) {\textbf{F}};
		\node at ($(g.north west) + (0.2,-0.2)$) {\textbf{G}};

	\end{tikzpicture}
	\caption{%
		\textbf{A second implementation of homeostatic regulation leads to similar results of bistable population activity on BrainScaleS-2.}
		Homeostatic regulation incorporating stochastic rounding promotes bistable population activity for low input rates $h$.
		\textbf{(A)} This choice of regulation likewise adjusts the population rate $\nu$ close to a target value (dashed line).
		\textbf{(B)} The connectivity slightly exceeds the homeostatic regulation with probabilistic update acceptance.
		\textbf{(C)} The distribution of weights $c_{ij}w_{ij}^\mathrm{rec}$ is comparable to the one reached with stochastic update acceptance.
		\textbf{(D)} However, the average weight $w^\mathrm{rec}$ is slightly lower.
		\textbf{(E)} The population activity $\nu$ features exponentially shaped autocorrelation functions $C(t')$ with autocorrelation times $\tau_\mathrm{AC}$ of same magnitude as the ones presented in main manuscript.
		\textbf{(F)} The distribution of the population activity $P(\nu)$ again resembles a bimodal trend for decreasing values of $h$, explaining the emergent autocorrelation times.
		\textbf{(G)} This bimodal trend is reflected in form of phases of high and low activity for low $h$.
	}
	\label{fig:hom_comparison}
\end{figure*}

\section{Comparison to a different implementation of homeostatic regulation \label{sec:hom_comparison}}

The neuromorphic implementation puts constraints on the range of implementable homeostatic regulation formulas.
As indicated by \cref{main-fig:stability}, the limited range, integer arithmetic of the \gls{ppu} requires additional mechanisms -- like the stochastic acceptance of weight updates -- to guarantee smooth convergence of the synaptic weights and in turn the attainment of the desired target rate.
While the stochastic acceptance renders the resulting weight distributions diverse (\cref{main-fig:chip}D), it likewise assigns many synapses a weight of zero (\cref{main-fig:chip}F).
To investigate whether the resulting bistable population activity is a result of our specific choice of homeostatic regulation, we implemented a second mechanism which aims to adapt the afferent synaptic weights of each neuron more uniformly.
Specifically, we utilize the same form as the one presented in \cref{main-eq:rule}, but with a smaller learning rate $\lambda = 0.39$ and instead of a stochastic acceptance probability added a small uniformly drawn random number to each update to realize stochastic rounding.
This mechanism likewise regulates the network activity (\cref{fig:hom_comparison} A to D), and, moreover, reveals emerging autocorrelations for low $h$ of similar magnitude (\cref{fig:hom_comparison}).
Different to the rule presented in the main manuscript, the connectivity increases monotonously with decreasing $h$ and only slightly decreases for very small $h$ again (\cref{fig:hom_comparison}B).
Again, these correlations stem from phases of low and high activity (\cref{fig:hom_comparison} F and G) in direct accordance with our previous results.


\section{Solution of the mean-field Langevin equation}

For the mean activity of a finite network, we obtain to leading order in system size
\begin{equation}\label{si_eq:MF_noise}
    \dot{\rho}(t) = h - a\rho(t) -b\rho^2(t) + \sigma\sqrt{\rho(t)/N}\eta(t)\, ,
\end{equation}
where $\eta(t)$ is Gaussian white noise with mean zero and unit variance and $\sigma/\sqrt{N}$ is a system-size dependent factor.
This (Ito) Langevin equation is equivalent to the Fokker Planck equation
\begin{align}\label{si_eq:FP_noise}
    \dot{P}(\rho,t) = &-\frac{\partial}{\partial\rho}\left[\left(h-a\rho(t)-b\rho^2(t)\right)P(\rho,t)\right]\nonumber\\
                    &+\frac{\sigma^2}{2N}\frac{\partial^2}{\partial\rho^2}\left[\rho(t)P(\rho,t)\right]\, .
\end{align}
Imposing $\dot{P}(A,t)=0$ and simplifying $f(\rho)=h-a\rho-b\rho^2$ as well as $g(\rho)=\rho\sigma^2/2N$, we get an equation for the stationary state $P^\ast(\rho)$
\begin{align}
    0
    &= -\frac{d}{d\rho}\left[f(\rho)P(\rho)\right] + \frac{d^2}{d\rho^2}\left[g(\rho)P(\rho)\right]\\
    &= \frac{d}{d\rho}\left(-f(\rho)P(\rho) +  \frac{d}{d\rho}\left[g(\rho)P(\rho)\right]\right) = \frac{d}{d\rho} j,
\end{align}
where $j$ reminds of the current in a continuity equation that has to be constant.
Imposing the zero-flux condition ($j=0$) one can derive the steady state probability distribution as usual~\cite{gardiner_handbook_1985}.
For this, we have to solve
\begin{align}
 \frac{d}{d\rho}\left[g(\rho)P(\rho)\right] = f(\rho)P(\rho).
\end{align}
Replacing $\Gamma(\rho)=g(\rho)P(\rho)$, we have to solve
\begin{align}
    \frac{d}{d\rho}\Gamma(\rho) = \Gamma(\rho)f(\rho)/g(\rho),
\end{align}
and obtain
\begin{align}
    \Gamma(\rho) = e^{\int \frac{f(\rho)}{g(\rho)}d\rho}.
\end{align}
In our case, $f(\rho)/g(\rho)= \left(h/\rho -a -b\rho\right)2N/\sigma^2$ such that
\begin{align}
    P(\rho) = \Gamma(\rho)/g(\rho) \propto \frac{2N}{\rho\sigma^2}e^{\frac{2N}{\sigma^2}\left(h\ln\rho-a\rho-\frac{b}{2}\rho^2\right)},
\end{align}
Rewriting $1/\rho=e^{-\ln{\rho}}$, we finally arrive at
\begin{equation}
    P(\rho) = \mathcal{N}\exp\left\{-\frac{2N}{\sigma^2}V(\rho)\right\}\, ,
\end{equation}
with a normalization constant $\mathcal{N}$ and the potential
\begin{align}
    V(\rho) = \left(\frac{\sigma^2}{2N}-h\right)\ln\rho + a\rho + \frac{b}{2}\rho^2\, .
\end{align}
